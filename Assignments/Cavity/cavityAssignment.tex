\documentclass[a4paper,english,12pt]{article}
\usepackage[latin1]{inputenc}
\usepackage[T1]{fontenc}
\usepackage{babel, verbatim, graphicx}

\begin{document}

\title{Assignment 1 - A 2D Cavity}
\author{}
\date{}
\maketitle


In this project we will study \textbf{(usikker p� bra fluidmekanisk beskrivelse)} using the open sourced software \textit{OpenFOAM}. To install OpenFOAM on Ubuntu, simply run the installation scripts provided on the course website. \textbf{(vet ikke hvordan du vil gj�re det)}

If you do not have an Ubuntu machine, you can find a tutorial on how to build OpenFOAM from source on (note: This is no simple matter)

\begin{verbatim}
http://www.openfoam.org/download/source.php
\end{verbatim}

Once you have completed the installation, run the tutorial script to build the tutorial enviroment. In this enviroment you will find several examples, i.e. \textit{Cavity}, located in the \textit{incompressible/isoFoam/cavity} folder. This example will be the main focus of this first assignment.

It is recommended that you use the openFOAM cavity tutorial in parallel with the assignment. The tutorial can be found at

\begin{verbatim}
http://www.openfoam.org/docs/user/cavity.php
\end{verbatim}

\section*{Task 1}
\subsection*{\textbf{a)}}

Build a mesh of a  $0.1\times0.1\times0.01$ (in units of meters) box resembling a 2D cavity. Use a movable top wall, fixed sidewalls and empty front and back walls.

\subsection*{\textbf{b)}}

Set boundary/initial conditions for \textit{p} and \textit{U} in the following manner:

\begin{itemize}
 \item \textbf{p:} $\nabla p = 0 m^2/s^2$ on fixedwalls and moving walls. Empty walls should be given an empty condition. The internal field should be zero.
 \item \textbf{U:} The internal field and sidewalls should be fixed zero. Front and back walls should be given an empty condition. The top wall should have a $1m/s$ condition in the x-direction, and zero elsewhere.
\end{itemize}

\vspace{0.5cm}
Use a kinematic viscosity which yields a Reynolds number of 10. Use a timestep of $0.005 s$ and setup your controlDict is such a manner that you store values every $0.1 s$ from $0-0.5 s$.

Run the \textit{icoFoam} application and visualize the pressure field at $t=0.5 s$ using paraFOAM.

\subsection*{\textbf{c)}}

\begin{itemize}
 \item Add a glyph filter displaying the interpolated pressure field on a slice of the $x,y$-plane.
 \item Use a stream tracer filter to visualize the same values. 
 \item Add a tube filter on top of the previous filter to get more significant lines.
\end{itemize}

\noindent
You should end up with something resembling Fig.~\ref{fig:streamLineTubeTask1}.

\begin{figure}[hbpt]
\centering
\includegraphics[scale=0.5]{StreamLineTubeTask1.jpg}
\caption{Plot of the stream lines of the cavity tutorial using the Stream Tracer and Tube filters.}
\label{fig:streamLineTubeTask1}
\end{figure}


\section*{Task 2}

Create a new cavity case, copying your current \textit{control} and \textit{system} folders into a new directory called \textit{CavityFine}. Double the amount of cells in the $x$ and $y$ directions and rebuild the mesh. Map the old data from $t=0 s$ to $t=0.5 s$ on the new grid using the \textit{mapFields} tool. 

\vspace{0.5cm}
\noindent
Configure the application to start from $t=0.5 s$ and end at $t=0.7 s$. Use a timestep of $0.0025 s$.

\vspace{0.5cm}
\noindent
Run the icoFoam-application and visualize the refined data.

\section*{Task 3}

Introduce mesh grading to the cavity tutorial in a new case called \textit{CavityGrade}. Create 4 blocks, each with 10 cells in the $x$ and $y$ directions. The ratio between the largest and smallest cells should be two.

\vspace{0.5cm}
\noindent
Change the time step to $2.5 ms$. Set the start time to $0.7 s$ and end time to $0.8 s$. Make sure your write interval is $0.1 s$ still.

\vspace{0.5cm}
\noindent
Map the CavityFine values to the new mesh, run the application and visualize the results.

\section*{Task 4}

Open all three cases simultaneously in paraFOAM. Plot the distance from the cavity base vs. the  x-component of the velocity field for all different cases. Superpose the images and comment on the result. In order to retrieve the required quantities, you should run

\begin{verbatim}
 foamCalc components U
\end{verbatim}

\noindent
in all the case folders. To open several simulations in paraFOAM simultaneously, you must first create an empty file called \textit{folderName.openFOAM} in the 'folderName' case folder. This empty file can be opened from paraFOAM.


\section*{Task 5}

Create a new case \textit{CavityHighRe}, and copy all the constituents of the original \textit{Cavity} folder into it. 

\vspace{0.5cm}
\noindent
Increase the Reynolds number to 100 by reducing the kinematic viscosity by a factor 10.

\vspace{0.5cm}
\noindent
Set the start time to $0.5 s$, the end time to $2 s$. 

\vspace{0.5cm}
\noindent
Run the application and compare the velocityfield at $t=2 s$ with the result from the original low Reynolds number Cavity case.


\section*{Task 6}

Run a simulation of a high Reynolds number case ($10^4$) using a RAS-model with the following initial conditions

\begin{itemize}
 \item U and p as in the old case,
 \item $k = 3.75\times 10^{-3} m^2/s^2,$
 \item $\epsilon = 7.65\times 10^{-4} m^2/s^3.$
\end{itemize}


Use a time step of $0.005 s$ and end at $20 s$ and visualize the results at consecutive time steps. Comment on the evolution.

\section*{Task 7}

Prepare a new simulation of a cavity where a square block of length $0.04 m$ is removed from the bottom right corner of the mesh. Edit the blockMesh file and build the new mesh. Copy the files from the \textit{0}-folder into a \textit{0.5}-folder in order to map incosistent data from the original Cavity case onto the new grid.

\vspace{0.5cm}
\noindent
Set up the \textit{cuttingPatches} and \textit{patchMap} dictionaries in the \textit{mapFields} dictionary to prepare for the incosistent maping. When you are done, run the \textit{mapFields} with th original Cavity case as a source.

\vspace{0.5cm}
\noindent
Visualize the fields. You will notice the solution around the clipped box does not suit the new case. Manually access these values and zero them out. Visualize the results again.




\end{document}

