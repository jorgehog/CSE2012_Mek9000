\documentclass[a4paper,english,12pt]{article}
\usepackage[latin1]{inputenc}
\usepackage[T1]{fontenc}
\usepackage{babel, verbatim, graphicx}

\begin{document}

\title{Assignment 2 - Channel flow using different LES models}
\author{}
\date{}
\maketitle


In this project we will study \textbf{(usikker p� bra fluidmekanisk beskrivelse)} using different Large Eddy Simulation models in openFOAM.

As a starting point you should use the case provided in the tutorials

\begin{verbatim}
/.../tutorials/incompressible/channelFoam/channel395
\end{verbatim}

Save the different LES model simulations each their own respective folders, so you can access to all the data for later comparisons.

\section*{One Equation Eddy}

The standard choice of LES model in the channelFoam tutorial is \textit{One Equation Eddy}. All properties of the LES model is set in the 

\begin{verbatim}
/.../constant/LESProperties 
\end{verbatim}

\noindent
dictionary. As a starting point you should open the file and familiarize with it. 

\subsection*{\textbf{a)}}

Before you start, take a look at the time steps etc. in the various files located in the \verb+control+ and \verb+system+ folders. Then build the mesh as you did in the previous assignment, and run the \verb+channelFoam+ application. The runtime of the simulation varies from machine to machine. On older laptops, the runtime can be as much as 2 hours on a single CPU. 

If you wish to run openFOAM in parallel, check out the following guide

\begin{verbatim}
www.openfoam.org/docs/user/running-applications-parallel.php
\end{verbatim}

\noindent
for details. The essence is to create a \verb+decomposePar+ file in the \verb+system+ folder, where you set the number of cores, decomposition method etc., before running the \verb+decomposePar+ program from your main directory. You can start the process by typing i.e. \verb+mpirun -np 4 channelFoam -paralell+ on a quad-core machine.

\subsection*{\textbf{b)}}

Open the simulation in paraFOAM. Select U, p and UMean\footnote{The zeroth time folder may not containt the value of UMean. If this is the case, select the \textit{Skip Zero Time} checkbox before selecting mesh parts and vector fields.} values, and make a slice in the middle of the channel with a $Z$-normal. Display the interpolated velocity field and play the simulation to the endpoint.

On the slice, apply a \textit{calculator} tool by either going through \vspace{0.5cm}\\\verb+filters+$\rightarrow$\verb+alphabetical+$\rightarrow$\verb+calculator+\\ 

\noindent
or press the calculator icon next to the slice-tool. Choose a reasonable name and select to calculate 

\begin{verbatim}
 U - UMean.
\end{verbatim}

Once you have applied the calculator tool, go to the display tab and select \textit{Colour by} the magnitude of the new vector created by the calculator. Replay the simulation.

\textbf{More turbulence exercises needs to be added}

\section*{Smagorinsky}

In order to switch LES model, you need to change the \verb+system/LESProperties+ file to

\begin{verbatim}
LESModel        Smagorinsky;
\end{verbatim}

\noindent
Retrace the steps of the One Equation Eddy section.

\newpage
\section*{Dynimic One Equation Eddy}

In order to change to this case we need something else besides a simple name change. Add the following to the \verb+LESProperties+ file:

\begin{verbatim}
LESModel        dynOneEqEddy;

...

dynOneEqEddyCoeffs
{
    ce               1.05;
    filter          simple;
}


\end{verbatim}

\noindent
Retrace the steps of the One Equation Eddy section.


\section*{Homogeneous Dynamic Smagorinsky}

Perform another switch by replacing the LESModel keyword and add necessary coefficients:

\begin{verbatim}
LESModel        homogeneousDynSmagorinsky;

...

homogeneousDynSmagorinskyCoeffs
{
    ce 1.048;
    filter simple;
}


\end{verbatim}

\noindent
Retrace the steps of the previous sections.



\section*{Dynamic Lagrangian}

Update the LESModel with coefficients as below:

\begin{verbatim}
LESModel        dynLagrangian;

...

dynLagrangianCoeffs
{
    filter simple;
    ce 1.048;
    theta 1.5;
}


\end{verbatim}

The dynamic Lagrangian model needs, in addition to coefficients, \textit{fmm} and \textit{flm} fields with initial conditions. Enter the \verb+0+ dictionary (after mesh is built) and add the files on the following pages.

\newpage
\tiny
\begin{verbatim}
fml:

/*--------------------------------*- C++ -*----------------------------------*\
| ========= | |
| \\ / F ield | OpenFOAM: The Open Source CFD Toolbox |
| \\ / O peration | Version: 2.0.0 |
| \\ / A nd | Web: www.OpenFOAM.com |
| \\/ M anipulation | |
\*---------------------------------------------------------------------------*/
FoamFile
{
version 2.0;
format ascii;
class volScalarField;
object flm;
}
// * * * * * * * * * * * * * * * * * * * * * * * * * * * * * * * * * * * * * //

dimensions [0 4 -4 0 0 0 0];

internalField uniform 0;

boundaryField
{
    bottomWall
    {
        type            fixedValue;
        value           uniform 0;
    }
    topWall
    {
        type            fixedValue;
        value           uniform 0;
    }
    sides1_half0
    {
        type            cyclic;
    }
    sides2_half0
    {
        type            cyclic;
    }
    inout1_half0
    {
        type            cyclic;
    }
    inout2_half0
    {
        type            cyclic;
    }
    sides2_half1
    {
        type            cyclic;
    }
    sides1_half1
    {
        type            cyclic;
    }
    inout1_half1
    {
        type            cyclic;
    }
    inout2_half1
    {
        type            cyclic;
    }
}

// ************************************************************************* //

\end{verbatim}
\newpage
\begin{verbatim}


fmm:

/*--------------------------------*- C++ -*----------------------------------*\
| ========= | |
| \\ / F ield | OpenFOAM: The Open Source CFD Toolbox |
| \\ / O peration | Version: 2.0.0 |
| \\ / A nd | Web: www.OpenFOAM.com |
| \\/ M anipulation | |
\*---------------------------------------------------------------------------*/
FoamFile
{
version 2.0;
format ascii;
class volScalarField;
object fmm;
}
// * * * * * * * * * * * * * * * * * * * * * * * * * * * * * * * * * * * * * //

dimensions [0 4 -4 0 0 0 0];

internalField uniform 1;

boundaryField
{
    bottomWall
    {
        type            fixedValue;
        value           uniform 1;
    }
    topWall
    {
        type            fixedValue;
        value           uniform 1;
    }
    sides1_half0
    {
        type            cyclic;
    }
    sides2_half0
    {
        type            cyclic;
    }
    inout1_half0
    {
        type            cyclic;
    }
    inout2_half0
    {
        type            cyclic;
    }
    sides2_half1
    {
        type            cyclic;
    }
    sides1_half1
    {
        type            cyclic;
    }
    inout1_half1
    {
        type            cyclic;
    }
    inout2_half1
    {
        type            cyclic;
    }
}

// ************************************************************************* // 
\end{verbatim}

\normalsize

\newpage

You will also need to provide solvers and schemes for doing calculations with these values. Enter the \verb+system/fvSolution+ dictionary and add the following solvers to the \verb+solvers+ dictionary

\begin{verbatim}
solvers
{
...

    flm
    {
        solver PBiCG;
        preconditioner DILU;
        tolerance 1e-05;
        relTol 0;
    }

    fmm
    {
        solver PBiCG;
        preconditioner DILU;
        tolerance 1e-05;
        relTol 0;
    }

...
}
\end{verbatim}

Now enter the \verb+system/fvSchemes+ and add the following to the \verb+divSchemes+ dictionary:

\begin{verbatim}
divSchemes
{
...

    div(phi,flm) Gauss limitedLinear 1;
    div(phi,fmm) Gauss limitedLinear 1;

...
}
\end{verbatim}

\noindent
You should now be able to run the application. Retrace the steps of the previous sections.


\newpage
\section*{Comparing results with DNS}

Open all the channelFoam cases in paraFOAM simultaneously. In order to do this, create blank files \verb+myFolderName.OpenFOAM+ by using i.e. the UNIX \textit{touch} command from the terminal:

\begin{verbatim}
touch myFolderName.OpenFOAM
\end{verbatim} 

You should now be able to select this file from the paraFOAM open tool. To split the display window into seperate parts, simply select the preferred split from the icons in the upper right corner of the display window. Pressing a window will select it. When a window is selected, you can toggle which parts you want to display by clicking the eye icons in the pipeline browser as in the previous assignment.

\textbf{NEED TO ADD SPESIFICS ABOUT THE COMPARISON}
\end{document}

