\documentclass[a4paper,english,12pt]{article}
\usepackage[latin1]{inputenc}
\usepackage[T1]{fontenc}
\usepackage{babel, verbatim, graphicx}

\begin{document}

\title{Assignment 2 - Channel flow using different LES models}
\author{}
\date{}
\maketitle


In this project we will study \textbf{(usikker p� bra fluidmekanisk beskrivelse)} using openFOAM using different Large Eddy Simulation (LES) models.

As a starting point you should use the 

\begin{verbatim}
/.../tutorials/incompressible/channelFoam/channel395
\end{verbatim}

\noindent
example.

For every new LES model you study, create a new folder structure for it. You are going to need the results from all models in order to compare the results later. 


\section*{One Equation Eddy}

The standard choice of LES model in the channelFoam tutorial is \textit{One Equation Eddy}. All properties of the LES model is set in the 

\begin{verbatim}
/.../constant/LESProperties 
\end{verbatim}

\noindent
dictionary. As a starting point you should open the file and familiarize with it. 

\subsection*{\textbf{a)}}

Before you start, take a look at the time steps etc. in the various files located in the \verb+control+ and \verb+system+ folders. Then build the mesh as you did in the previous assignment, and run the \verb+channelFoam+ application. The runtime of the simulation varies from machine to machine. On older laptops, the runtime can be as much as 2 hours on a single CPU. 

If you wish to run openFOAM in parallel, check out the following guide

\begin{verbatim}
www.openfoam.org/docs/user/running-applications-parallel.php
\end{verbatim}

\noindent
for details.

\subsection*{\textbf{b)}}

Open the simulation in paraFOAM. Select U, p and UMean values, and make a slice in the middle of the channel with a $Z$-normal, choose to display the interpolated velocity field and play the simulation to the endpoint.

On the slice, apply a \textit{calculator} tool by either going through \vspace{0.5cm}\\\verb+filters+$\rightarrow$\verb+alphabetical+$\rightarrow$\verb+calculator+\\ 
\vspace{0.5cm} 
\noindent
or press the calculator icon next to the slice-tool. Choose a reasonable name and select to calculate 

\begin{verbatim}
 U - UMean.
\end{verbatim}

Once you have applied the calculator tool, go to the display tab and select \textit{Colour by} the name of the value you created with the calculator. Select magitude and play the simulation again\footnote{You might run into problems with segmentation faults at this point. OpenFOAM does not store the averages for the zeroth time. If this is a problem, at the initial properties tab, select \textit{Skip Zero Time}, move the simulation to the first time step, and then select the preferred values for the visualization.}.

\textbf{More exercises needs to be added}

\section*{Smagorinsky}

In order to do the switching you need to change the \verb+LESProperties+ file

\begin{verbatim}
LESModel        Smagorinsky;
\end{verbatim}

\noindent
Retrace the steps of the One Equation Eddy section.

\newpage
\section*{Dynimic One Equation Eddy}

In order to change to this case we need something else besides a simple name change. Add the following to the \verb+LESProperties+ file:

\begin{verbatim}
LESModel        dynOneEqEddy;

...

dynOneEqEddyCoeffs
{
    ce               1.05;
    filter          simple;
}

need the one eq eddy coeffs too?

\end{verbatim}

\noindent
Retrace the steps of the One Equation Eddy section.


\section*{Homogeneous Dynimic Smagorinsky}

Add the following to the \verb+LESProperties+ file:

\begin{verbatim}
LESModel        homogeneousDynSmagorinsky;

...

homogeneousDynSmagorinskyCoeffs
{
    ce 1.048;
    filter simple;
}

the same as dynOneEqEddy??

\end{verbatim}

\noindent
Retrace the steps of the previous sections.



\section*{Dynamic Lagrangian}

This time we need to give additional information beyond the single file. Add the following to the \verb+LESProperties+ file:

\begin{verbatim}
LESModel        dynLagrangian;

...

dynLagrangianCoeffs
{
    filter simple;
    ce 1.048;
    theta 1.5;
}

CE same for all?

\end{verbatim}

In addition you will need some initial conditions for the \verb+fmm+ and \verb+flm+ values. Enter the \verb+0+ dictionary (after mesh is built) and add the files on the following pages.

\newpage
\tiny
\begin{verbatim}
fml:

/*--------------------------------*- C++ -*----------------------------------*\
| ========= | |
| \\ / F ield | OpenFOAM: The Open Source CFD Toolbox |
| \\ / O peration | Version: 2.0.0 |
| \\ / A nd | Web: www.OpenFOAM.com |
| \\/ M anipulation | |
\*---------------------------------------------------------------------------*/
FoamFile
{
version 2.0;
format ascii;
class volScalarField;
object flm;
}
// * * * * * * * * * * * * * * * * * * * * * * * * * * * * * * * * * * * * * //

dimensions [0 4 -4 0 0 0 0];

internalField uniform 0;

boundaryField
{
    bottomWall
    {
        type            fixedValue;
        value           uniform 0;
    }
    topWall
    {
        type            fixedValue;
        value           uniform 0;
    }
    sides1_half0
    {
        type            cyclic;
    }
    sides2_half0
    {
        type            cyclic;
    }
    inout1_half0
    {
        type            cyclic;
    }
    inout2_half0
    {
        type            cyclic;
    }
    sides2_half1
    {
        type            cyclic;
    }
    sides1_half1
    {
        type            cyclic;
    }
    inout1_half1
    {
        type            cyclic;
    }
    inout2_half1
    {
        type            cyclic;
    }
}

// ************************************************** *********************** //

\end{verbatim}
\newpage
\begin{verbatim}


fmm:

/*--------------------------------*- C++ -*----------------------------------*\
| ========= | |
| \\ / F ield | OpenFOAM: The Open Source CFD Toolbox |
| \\ / O peration | Version: 2.0.0 |
| \\ / A nd | Web: www.OpenFOAM.com |
| \\/ M anipulation | |
\*---------------------------------------------------------------------------*/
FoamFile
{
version 2.0;
format ascii;
class volScalarField;
object fmm;
}
// * * * * * * * * * * * * * * * * * * * * * * * * * * * * * * * * * * * * * //

dimensions [0 4 -4 0 0 0 0];

internalField uniform 1;

boundaryField
{
    bottomWall
    {
        type            fixedValue;
        value           uniform 1;
    }
    topWall
    {
        type            fixedValue;
        value           uniform 1;
    }
    sides1_half0
    {
        type            cyclic;
    }
    sides2_half0
    {
        type            cyclic;
    }
    inout1_half0
    {
        type            cyclic;
    }
    inout2_half0
    {
        type            cyclic;
    }
    sides2_half1
    {
        type            cyclic;
    }
    sides1_half1
    {
        type            cyclic;
    }
    inout1_half1
    {
        type            cyclic;
    }
    inout2_half1
    {
        type            cyclic;
    }
}

// ************************************************** *********************** // 
\end{verbatim}

\normalsize

\noindent
You should now be able to run the application. Retrace the steps of the previous sections.


\newpage
\section*{Comparing results with DNS}

\textbf{Need more info}

...
\end{document}

